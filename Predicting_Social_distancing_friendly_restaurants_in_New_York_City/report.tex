\documentclass[11pt]{article}

\usepackage[top=0.5in, bottom=0.5in, left=0.5in, right=0.5in]{geometry}
\usepackage{authblk}
\usepackage{hyperref}
\usepackage[utf8]{inputenc}
\usepackage{amsmath}
\usepackage{amsfonts}
\usepackage{amssymb}
\usepackage{siunitx}
\usepackage{graphicx}
\usepackage{subcaption}
\usepackage{float}
\usepackage[nottoc,numbib]{tocbibind}
\usepackage{biblatex}
\usepackage{parskip}

\bibliography{references.bib}

\title{Predicting Social distancing friendly restaurants in New York City}
\author{Aanchal Narendran}

\makeatletter
\let\inserttitle\@title
\let\insertauthor\@author
\makeatother

\begin{document}

\begin{center}
  \LARGE{\inserttitle}

  \Large{\insertauthor}
\end{center}

\section{Introduction}

\par 
The restaurant business in New York City is unlike other business in the world. With 50,133 eating or drinking locations \cite{restaurants_statistics} and employing over 684,100 people \cite{restaurants_statistics} , The restaurant industry is vital to the economic and social fabric of New York City. Restaurants also aid in the development of local communities throughout the state. The cuisine of New York city comprises many cuisines belonging to many ethnic groups. During the period of 2013-2016, approximately 37 percent of adults consumed fast food on a given day \cite{fastfood_percent:paper}. New York being home to numerous universities whilst being one of the financial capitals of the world caters to both alike. Studies have shown that consumption of food eaten away from home has also risen alarmingly. 
\par As of March 17, all restaurants in New York City have been banned from serving customers in house and many have completely closed doors. These restaurants chose to fall back on delivery apps such as Grubhub and kept their institutions running with skeleton crews. The spending in restaurants was down by 90 percent in late March, as opposed to the same time period last year \cite{NewYorkBudget:paper}. As New York City is entering the phase 3 of it's reopening, The number of employees at work place would be higher increasing the necessity for food being outsourced. 
\par The project aims to predict the most low risk restaurants in a certain neighborhood. This criteria has been determined by the center for disease control and prevention. It will prompt the user to input the neighborhood they are currently situated in based on which it would display the most low risk restaurants. 

\section{Data}
\par For this project, We will majorly be using 4 datasets/data sources. They are as follows:
\begin{enumerate}
    \item Foursquare places API: Extract names of restaurants and their details
    \item New York JSON: Contains all neighbourhood and boroughs as well as their coordinates
    \item Cases by Borough: Contains list of active number of cases in a borough
    \item DOHMH New York City Restaurant Inspection Results: Contains the recent grade the restaurant received during an inspection
\end{enumerate}
\par The Foursquare API has a list of all the restaurants in the neighborhood. For now, the User shall only be allowed to specify the neighbourhood. The Venue details endpoint is used to extract various characteristics of the venue. The Venue details endpoint is used to extract various characteristics of the venue. The venue details endpoint contains data such as whether the location offers delivery, has outside seating and a drivethrough. According to CDC, Restaurants which have drivethroughs, delivery, take out and curb side pickup are low risk restaurants\cite{restaurantconsiderations:article}. Outside seating makes it a comparatively more risk restaurant. Inside seating is even more risk.
\par The active number of cases would work as a cutoff. In the end result, if the active number of cases is greater than a certain threshold value, It would prompt the user to stick to delivery options. The Inspection results are being added to ensure additional assurance of hygienic practices in the restaurant. 
\printbibliography
\end{document}
